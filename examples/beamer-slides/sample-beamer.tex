\documentclass[aspectratio=169, 11pt]{beamer}

% ==============================================================================
% Self-contained Beamer example — no external Preambles/ dependency
% ==============================================================================

\usetheme{metropolis}
\usepackage{amsmath,amssymb}
\usepackage{booktabs}
\usepackage{xcolor}
\usepackage{tcolorbox}
\tcbuselibrary{skins, breakable}

% --- Colors ---
\definecolor{primaryblue}{HTML}{2C3E50}
\definecolor{primarygold}{HTML}{D4A843}
\definecolor{highlightyellow}{HTML}{F39C12}
\definecolor{lightbg}{HTML}{F2F1EF}
\definecolor{methodblue}{HTML}{34495E}

\setbeamercolor{frametitle}{bg=primaryblue, fg=white}
\setbeamercolor{title}{fg=primaryblue}
\setbeamercolor{subtitle}{fg=primarygold}
\setbeamercolor{structure}{fg=primaryblue}
\setbeamercolor{itemize item}{fg=primaryblue}
\setbeamercolor{itemize subitem}{fg=primarygold}

% --- Custom Environments ---
\newenvironment{keybox}{%
  \begin{tcolorbox}[colback=primarygold!12, colframe=primarygold,
    leftrule=4pt, rightrule=0pt, toprule=0pt, bottomrule=0pt,
    arc=3pt, boxrule=0pt]
}{%
  \end{tcolorbox}
}

\newenvironment{highlightbox}{%
  \begin{tcolorbox}[colback=highlightyellow!8, colframe=highlightyellow,
    leftrule=4pt, rightrule=0pt, toprule=0pt, bottomrule=0pt,
    arc=3pt, boxrule=0pt]
}{%
  \end{tcolorbox}
}

\newenvironment{methodbox}{%
  \begin{tcolorbox}[colback=primaryblue!6, colframe=primaryblue,
    leftrule=4pt, rightrule=0pt, toprule=0pt, bottomrule=0pt,
    arc=3pt, boxrule=0pt]
}{%
  \end{tcolorbox}
}

\newtcolorbox{definitionbox}[1][]{%
  colback=white, colframe=primaryblue,
  arc=3pt, boxrule=1.5pt,
  title={\textbf{#1}},
  coltitle=primaryblue, fonttitle=\bfseries,
  attach boxed title to top left={yshift=-2mm, xshift=5mm},
  boxed title style={colback=white, colframe=white},
  nobeforeafter
}

% --- Metadata ---
\title{Sample Beamer Deck}
\subtitle{Custom Environment Showcase}
\author{Academic Author}
\institute{Department of Economics}
\date{\today}

\begin{document}

% ==============================================================================
\begin{frame}
\titlepage
\end{frame}

% ==============================================================================
\begin{frame}{Itemize and Enumerate}

Key concepts in causal inference:

\begin{itemize}
  \item \textbf{Potential outcomes framework} --- each unit has outcomes under treatment and control
  \item \textbf{Fundamental problem} --- we only observe one potential outcome per unit
    \begin{itemize}
      \item Selection bias arises when treatment is not randomly assigned
      \item Randomization solves this by balancing confounders in expectation
    \end{itemize}
  \item \textbf{Identification strategies} --- exploit natural variation or design
\end{itemize}

\vfill

\begin{enumerate}
  \item Formulate a research question
  \item Identify a credible source of variation
  \item Estimate and interpret with care
\end{enumerate}

\end{frame}

% ==============================================================================
\begin{frame}{Mathematics}

\textbf{Theorem (Frisch--Waugh--Lovell).}
Consider the regression $Y = X_1 \beta_1 + X_2 \beta_2 + \varepsilon$. Then $\hat{\beta}_1$ can be obtained by:

$$
\hat{\beta}_1 = (X_1' M_2 X_1)^{-1} X_1' M_2 Y
$$

where $M_2 = I - X_2(X_2'X_2)^{-1}X_2'$ is the annihilator matrix.

\vspace{0.5em}

\textit{Proof sketch.} Partial out $X_2$ from both $Y$ and $X_1$, then run OLS on the residuals. The coefficient on the residualized $X_1$ equals $\hat{\beta}_1$. \qed

\end{frame}

% ==============================================================================
\begin{frame}{Custom Box Environments}

\begin{keybox}
\textbf{Key Result:} Under parallel trends, the DiD estimator is unbiased for the ATT with $\hat{\tau} = (\bar{Y}_{T,post} - \bar{Y}_{T,pre}) - (\bar{Y}_{C,post} - \bar{Y}_{C,pre})$.
\end{keybox}

\begin{highlightbox}
\textbf{Highlight:} Staggered adoption requires heterogeneity-robust estimators such as Callaway and Sant'Anna (2021).
\end{highlightbox}

\begin{methodbox}
\textbf{Method:} Estimate using TWFE with standard errors clustered at the state level. Use 50 bootstrap replications for inference.
\end{methodbox}

\end{frame}

% ==============================================================================
\begin{frame}{Definition Box}

\begin{definitionbox}[Average Treatment Effect]
The \textbf{ATE} is defined as:
$$
\tau_{ATE} = E[Y_i(1) - Y_i(0)]
$$
where $Y_i(1)$ and $Y_i(0)$ are the potential outcomes under treatment and control, respectively.
\end{definitionbox}

\vspace{0.5em}

\begin{definitionbox}[Average Treatment Effect on the Treated]
The \textbf{ATT} conditions on receiving treatment:
$$
\tau_{ATT} = E[Y_i(1) - Y_i(0) \mid D_i = 1]
$$
\end{definitionbox}

\end{frame}

% ==============================================================================
\begin{frame}{Table}

\begin{table}
\centering
\caption{Regression Results}
\begin{tabular}{lcc}
\toprule
 & (1) OLS & (2) IV \\
\midrule
Treatment    & 0.152***  & 0.238** \\
             & (0.041)   & (0.097) \\
Controls     & Yes       & Yes     \\
Fixed Effects & Unit, Time & Unit, Time \\
\midrule
Observations & 5{,}000   & 5{,}000 \\
$R^2$        & 0.43      & ---     \\
F-stat (1st) & ---       & 24.7    \\
\bottomrule
\multicolumn{3}{l}{\footnotesize $^{***}p<0.01$, $^{**}p<0.05$, $^{*}p<0.1$. SEs clustered by unit.}
\end{tabular}
\end{table}

\end{frame}

% ==============================================================================
\begin{frame}{Two-Column Layout}

\begin{columns}[T]
\begin{column}{0.48\textwidth}
\textbf{Advantages of DiD}
\begin{itemize}
  \item Intuitive and transparent
  \item Controls for time-invariant confounders
  \item Works with observational data
  \item Widely applicable
\end{itemize}
\end{column}

\begin{column}{0.48\textwidth}
\textbf{Limitations}
\begin{itemize}
  \item Requires parallel trends
  \item Sensitive to functional form
  \item Cannot handle time-varying confounders
  \item Staggered timing complications
\end{itemize}
\end{column}
\end{columns}

\end{frame}

\end{document}